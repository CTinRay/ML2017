\documentclass[fleqn,a4paper,12pt]{article}
\usepackage[top=1in, bottom=1in, left=1in, right=1in]{geometry}



\title{Machine Learning Homework 5}
\date{}

\setcounter{section}{0}

\usepackage{listings}

\usepackage{amsmath}
\usepackage{amssymb}


\usepackage{mathspec}
\setmainfont{Noto Serif CJK TC}
% \setmathsfont(Digits,Latin,Greek)[Numbers={Lining,Proportional}]{DejaVu Math TeX Gyre}
\newfontfamily\ZhFont{Noto Serif CJK TC}
\newfontfamily\SmallFont[Scale=0.8]{Droid Sans}
% \newfontfamily\SmallSmallFont[Scale=0.7]{Noto Serif CJK}
\usepackage{fancyhdr}
\usepackage{lastpage}
\pagestyle{fancy}
\fancyhf{}
\rhead{B03902072\ZhFont{江廷睿}}
\lhead{Machine Learning Homework 4}
\rfoot{\thepage / \pageref{LastPage}}

\XeTeXlinebreaklocale "zh"
\XeTeXlinebreakskip = 0pt plus 1pt
\usepackage{parskip}

\usepackage{graphicx}
\usepackage{caption}
\usepackage{subcaption}
\usepackage{float}

\begin{document}
% \maketitle
\thispagestyle{fancy}

\section{(1\%)請問softmax適不適合作為本次作業的output layer? 寫出你最後選擇的output layer並說明理由。}
\section{(1\%)請設計實驗驗證上述推論}
\section{(1\%)請試著分析tags的分布情況(數量)}

\begin{tabular}{| l | c | r |}
  AUTOBIOGRAPHICAL-NOVEL & 31 \\
  DETECTIVE-FICTION & 178 \\
  WAR-NOVEL & 31 \\
  NOVELLA & 29 \\
  NON-FICTION & 102 \\
  CHILDREN'S-LITERATURE & 777 \\
  HISTORICAL-FICTION & 137 \\
  SUSPENSE & 318 \\
  GOTHIC-FICTION & 12 \\
  ROMANCE-NOVEL & 157 \\
  HIGH-FANTASY & 15 \\
  HORROR & 192 \\
  CRIME-FICTION & 368 \\
  TECHNO-THRILLER & 18 \\
  ADVENTURE-NOVEL & 109 \\
  DYSTOPIA & 30 \\
  AUTOBIOGRAPHY & 51 \\
  COMIC-NOVEL & 37 \\
  ALTERNATE-HISTORY & 72 \\
  UTOPIAN-AND-DYSTOPIAN-FICTION & 11 \\
  HISTORY & 40 \\
  NOVEL & 992 \\
  SPECULATIVE-FICTION & 1448 \\
  MEMOIR & 35 \\
  THRILLER & 243 \\
  SHORT-STORY & 41 \\
  MYSTERY & 642 \\
  APOCALYPTIC-AND-POST-APOCALYPTIC-FICTION & 14 \\
  SCIENCE-FICTION & 959 \\
  HUMOUR & 18 \\
  SPY-FICTION & 75 \\
  FICTION & 1672 \\
  YOUNG-ADULT-LITERATURE & 288 \\
  BIOGRAPHY & 42 \\
  FANTASY & 773 \\
  COMEDY & 59 \\
  HISTORICAL-NOVEL & 222 \\
  SATIRE & 35 \\
\end{tabular}

\section{(1\%)本次作業中使用何種方式得到word embedding?請簡單描述做法。}

使用 GloVe 已經事先用 Wikipedia 2014 + Gigaword 5 訓練好的 100 維資料。 GloVe 的訓練方式是先統計詞彙跟詞彙同時出現次數,計算出一個詞彙出現在另一詞彙的上下文的條件機率,並將這些機率寫成一個矩陣。然後再使用矩陣分解的方式,找出能最小化兩詞彙內積與前述的條件機率之誤差的向量表示法,即得到詞彙們的向量。

\section{(1\%)試比較bag of word和RNN何者在本次作業中效果較好。}


\end{document}
